Organize this section according to the rules defined in the project description. \\ 

\subsection{External Interface Requirements}

\subsubsection{User Interfaces}
Here will go wireframes
\subsubsection{Hardware Interfaces}
Both users and store managers can use the application through a mobile phone or a personal computer. Users unable to do so will use totems provided by stores.
\subsubsection{Software Interfaces}
map API\\
send ticket/booking requests to CLup\\
send turnstile entrances/exits to CLup\\
query automatically generated reports and obtain their statistics\\
\subsubsection{Communication Interfaces}
The only type of communication required by CLup is a stable internet connection.
\subsection{Functional Requirements}

\subsubsection{List of requirements}
\begin{table}[H]
\label{tab: ReqList}
\begin{tabular}{l|l}
	R1 & Every user can generate a quick ticket for any store \\
	R2 & \pbox{13cm}{Whenever user makes initiates a booking procedure, CLup must be able to compute a suggested least crowded time slot based on historical data} \\
	R3 & \pbox{13cm}{CLup must elaborate and upload data about current global customer affluence to the store during use} \\
	R4 & CLup must admit only valid QR codes for entrance \\
	R5 & CLup must allow users to know current queue status \\
	R6 & CLup must update user on tickets' validity change \\
	R7 & CLup must inform offline users about new tickets (un)availability \\
	R8 & CLup must allow users to indicate which product category they are going to purchase while booking \\
	R9 & CLup must suggest alternative stores when the combination of selected store|time gives no results \\
	R10 & CLup must reserve a non null number of paper tickets at any time for offline customers use 	\\
	R11 & CLup must gather all stores' data about entrance fluxes \\
	R12 & CLup is able to cross affluence data of any supermarket\\
	R13 & CLup keeps track of people who book an entrance and don’t come\\
	R14 & CLup allows store managers to stop quick tickets availability \\
	R15 & CLup is able to generate QR codes\\
	R16 & CLup is able to authenticate users\\
	R17 & CLup is able to store users’ data \\
	R18 & CLup is able to process users' data \\
	R19 & CLup makes quick ticket invalid after 15 minutes delay\\
	R20 & CLup can use stores’ data to sort every store by crowdedness \\
	R21 & Users can see available day/time slots of a supermarket through CLup\\
	R22 & CLup shows to store managers flux data about their supermarket (forse questo è quello che intendeva R3)\\
	R23 & CLup must be able to process reservations\\
	
\end{tabular}
\caption{Requirements list}
\end{table}

\subsubsection{Mapping}


\subsubsection{Use cases}

\begin{enumerate}
	
%TABELLA 1
\item \textbf{Registration of new account}\\\\

\begin{table}[H]
{\rowcolors{1}{white}{white}
\begin{tabular}{|c|p{14cm}|}
	\hline
	Name & Registration of new account\\
	\hline
	Actors & User\\
	\hline
	Entry Condition & User installed and opened the app and doesn’t have an account or wants to register another one\\
	\hline
	Event Flow & \begin{enumerate}
		\item User opens the app.
		\item Login screen loads.
		\item User taps “Sign up".
		\item TODO REGISTRATION WIREFRAME.
		\end{enumerate}\\
	\hline
	Exit Conditions & User now has an account with which he can log in\\
	\hline
	Exception & \begin{enumerate}
		\item There is no internet when the user presses “create account”
	\end{enumerate}
	
	“No internet” popup, the user can either wait for internet to come back or discard the incomplete account creation by going back to main screen\\
	\hline
\end{tabular}
}
\end{table}	

%TABELLA 2
\item \textbf{User login}\\\\
	
\begin{table}[H]
{\rowcolors{1}{white}{white}
	\begin{tabular}{|c|p{14cm}|}
		\hline
		Name & User login\\
		\hline
		Actors & User\\
		\hline
		Entry Condition & User already has an account\\
		\hline
		Event Flow & \begin{enumerate}
			\item User opens the app.
			\item Login screen loads.
			\item User already has an account.
			\item User inputs his credentials and presses “login”
			\item Main screen loads
		\end{enumerate}\\
		\hline
		Exit Conditions & User logged in and is now in main screen, from where he can virtually access all of the app’s functionalities\\
		\hline
		Exception & \begin{enumerate}
			\item Credentials are wrong\newline
			wrong credentials popup, user stays in  login screen
			
			\item There is no internet connection\newline
			no internet warning pop up, user still logs in if he used previously inserted and saved credentials so he can still edit settings and filters or look at his history 
			
		\end{enumerate}\\
		
		\hline
	\end{tabular}
}
\end{table}

%TABELLA 3
\item \textbf{Quick ticket request}\\\\
	
\begin{table}[H]
{\rowcolors{1}{white}{white}
	\begin{tabular}{|c|p{14cm}|}
		\hline
		Name & ASAP ticket request\\
		\hline
		Actors & User\\
		\hline
		Entry Condition & User successfully logged in and is in main screen\\
		\hline
		
		Event Flow & \begin{enumerate}
			\item User taps “Show stores”
			\item User is taken to “Quick results” screen
			\item IF LIST MODE (from settings)\newline
			stores with queue time, size and distance are shown in a list according to setted filters, if show more is pressed more stores are loaded and the list is made scrollable, if location button is pressed a map showing the store’s location is shown (TODO browse stores on map)\newline
			IF MAP MODE (from settings\newline)
			stores with queue time, size and distance are shown on a map, if list button is pressed User is taken to the previously descripted “list mode” case
			\item User taps on a store
			\item Confirmation pop up is shown
			\item If user confirms “Ticket screen” is shown, otherwise he is taken back to 3
		\end{enumerate}\\
	
		\hline
		Exit Conditions & User has a ticket with updating due time to enter the store\\
		\hline
		
		Exception & \begin{enumerate}
			\item The store blocked ticket requests\newline
			after 5 user is told that the store is no longer available and remains in list/map to make an eventual different choice
			
			\item There is no internet connection\newline
			after 1 user stays in main screen with a dismissable “no internet” pop up
			
			\item User already has an ASAP ticket\newline
			after 1 user stays in main screen with a dismissable “you can only have one ASAP ticket” 
			
		\end{enumerate}\\
		
		\hline
	\end{tabular}
}
\end{table}

%TABELLA 4
\item \textbf{Quick ticket request at physical Totem}\\\\

\begin{table}[H]
{\rowcolors{1}{white}{white}
	\begin{tabular}{|c|p{14cm}|}
		\hline
		Name & ASAP ticket request at physical Totem\\
		\hline
		Actors & User\\
		\hline
		Entry Condition & User starts interacting with CLup tablet (?) outside the store\\
		\hline
		
		Event Flow & \begin{enumerate}
			\item POSSIBLE IDENTIFICATION (fiscal code or ID card)
			\item User sees current queue for this store and decides whether to confirm or cancel
			\item A card ticket with remainder and QR is printed
			\item User leaves the station with his printed ticket
			
		\end{enumerate}\\
		
		\hline
		Exit Conditions & User has a ticket that he can scan to enter the store when he comes back at the written date and time\\
		\hline
		
		Exception & \begin{enumerate}
			\item The store blocked ticket requests\newline
			User can’t get a ticket and is asked to come back another time
			
		\end{enumerate}\\
		
		\hline
	\end{tabular}
}
\end{table}

%TABELLA 5
\item \textbf{Edit filters}\\\\

\begin{table}[H]
{\rowcolors{1}{white}{white}
	\begin{tabular}{|c|p{14cm}|}
		\hline
		Name & Edit filters\\
		\hline
		Actors & User\\
		\hline
		Entry Condition & User pressed filter button from main screen\\
		\hline
		
		Event Flow & \begin{enumerate}
			\item User is in main screen and taps the “filters” button
			\item User is in filters screen
			\item User changes the filter parameters he wants to change among:
			\begin{enumerate}
				\item distance range
				\item store type
				\item default booking time (ignored by ticket request)
				\item default calendar or stores first when booking (ignored by ticket request)
				\item default map or list view when booking (ignored by ticket request)
				\item ???
			\end{enumerate}
			
			\item User presses back to main screen button
			\item Popup to confirm and save or discard the new filters is shown 

		\end{enumerate}\\
		
		\hline
		Exit Conditions & New filters are set and they will affect the next ticket request or visit plan\\
		\hline
	
		Exception & \begin{enumerate}
			\item User closes the app without saving the filters\newline
			Filter modifications are lost
			
		\end{enumerate}\\
		
		\hline
	\end{tabular}
}
\end{table}

%TABELLA 6 TODO da copiare/incollare i vari campi!
\item \textbf{Edit filters}\\\\

\begin{table}[H]
{\rowcolors{1}{white}{white}
	\begin{tabular}{|c|p{14cm}|}
		\hline
		Name & Edit filters\\
		\hline
		Actors & User\\
		\hline
		Entry Condition & User pressed filter button from main screen\\
		\hline
		
		Event Flow & \begin{enumerate}
			\item User is in main screen and taps the “filters” button
			\item User is in filters screen
			\item User changes the filter parameters he wants to change among:
			\begin{enumerate}
				\item distance range
				\item store type
				\item default booking time (ignored by ticket request)
				\item default calendar or stores first when booking (ignored by ticket request)
				\item default map or list view when booking (ignored by ticket request)
				\item ???
			\end{enumerate}
			
			\item User presses back to main screen button
			\item Popup to confirm and save or discard the new filters is shown 
			
		\end{enumerate}\\
		
		\hline
		Exit Conditions & New filters are set and they will affect the next ticket request or visit plan\\
		\hline
		
		Exception & \begin{enumerate}
			\item User closes the app without saving the filters\newline
			Filter modifications are lost
			
		\end{enumerate}\\
		
		\hline
	\end{tabular}
}
\end{table}

%TABELLA 7
\item \textbf{Edit filters}\\\\

\begin{table}[H]
	{\rowcolors{1}{white}{white}
		\begin{tabular}{|c|p{14cm}|}
			\hline
			Name & Edit filters\\
			\hline
			Actors & User\\
			\hline
			Entry Condition & User pressed filter button from main screen\\
			\hline
			
			Event Flow & \begin{enumerate}
				\item User is in main screen and taps the “filters” button
				\item User is in filters screen
				\item User changes the filter parameters he wants to change among:
				\begin{enumerate}
					\item distance range
					\item store type
					\item default booking time (ignored by ticket request)
					\item default calendar or stores first when booking (ignored by ticket request)
					\item default map or list view when booking (ignored by ticket request)
					\item ???
				\end{enumerate}
				
				\item User presses back to main screen button
				\item Popup to confirm and save or discard the new filters is shown 
				
			\end{enumerate}\\
			
			\hline
			Exit Conditions & New filters are set and they will affect the next ticket request or visit plan\\
			\hline
			
			Exception & \begin{enumerate}
				\item User closes the app without saving the filters\newline
				Filter modifications are lost
				
			\end{enumerate}\\
			
			\hline
		\end{tabular}
	}
\end{table}

%TABELLA 8
\item \textbf{Edit filters}\\\\

\begin{table}[H]
	{\rowcolors{1}{white}{white}
		\begin{tabular}{|c|p{14cm}|}
			\hline
			Name & Edit filters\\
			\hline
			Actors & User\\
			\hline
			Entry Condition & User pressed filter button from main screen\\
			\hline
			
			Event Flow & \begin{enumerate}
				\item User is in main screen and taps the “filters” button
				\item User is in filters screen
				\item User changes the filter parameters he wants to change among:
				\begin{enumerate}
					\item distance range
					\item store type
					\item default booking time (ignored by ticket request)
					\item default calendar or stores first when booking (ignored by ticket request)
					\item default map or list view when booking (ignored by ticket request)
					\item ???
				\end{enumerate}
				
				\item User presses back to main screen button
				\item Popup to confirm and save or discard the new filters is shown 
				
			\end{enumerate}\\
			
			\hline
			Exit Conditions & New filters are set and they will affect the next ticket request or visit plan\\
			\hline
			
			Exception & \begin{enumerate}
				\item User closes the app without saving the filters\newline
				Filter modifications are lost
				
			\end{enumerate}\\
			
			\hline
		\end{tabular}
	}
\end{table}

%TABELLA 9
\item \textbf{Edit filters}\\\\

\begin{table}[H]
	{\rowcolors{1}{white}{white}
		\begin{tabular}{|c|p{14cm}|}
			\hline
			Name & Edit filters\\
			\hline
			Actors & User\\
			\hline
			Entry Condition & User pressed filter button from main screen\\
			\hline
			
			Event Flow & \begin{enumerate}
				\item User is in main screen and taps the “filters” button
				\item User is in filters screen
				\item User changes the filter parameters he wants to change among:
				\begin{enumerate}
					\item distance range
					\item store type
					\item default booking time (ignored by ticket request)
					\item default calendar or stores first when booking (ignored by ticket request)
					\item default map or list view when booking (ignored by ticket request)
					\item ???
				\end{enumerate}
				
				\item User presses back to main screen button
				\item Popup to confirm and save or discard the new filters is shown 
				
			\end{enumerate}\\
			
			\hline
			Exit Conditions & New filters are set and they will affect the next ticket request or visit plan\\
			\hline
			
			Exception & \begin{enumerate}
				\item User closes the app without saving the filters\newline
				Filter modifications are lost
				
			\end{enumerate}\\
			
			\hline
		\end{tabular}
	}
\end{table}

%TABELLA 10
\item \textbf{Edit filters}\\\\

\begin{table}[H]
	{\rowcolors{1}{white}{white}
		\begin{tabular}{|c|p{14cm}|}
			\hline
			Name & Edit filters\\
			\hline
			Actors & User\\
			\hline
			Entry Condition & User pressed filter button from main screen\\
			\hline
			
			Event Flow & \begin{enumerate}
				\item User is in main screen and taps the “filters” button
				\item User is in filters screen
				\item User changes the filter parameters he wants to change among:
				\begin{enumerate}
					\item distance range
					\item store type
					\item default booking time (ignored by ticket request)
					\item default calendar or stores first when booking (ignored by ticket request)
					\item default map or list view when booking (ignored by ticket request)
					\item ???
				\end{enumerate}
				
				\item User presses back to main screen button
				\item Popup to confirm and save or discard the new filters is shown 
				
			\end{enumerate}\\
			
			\hline
			Exit Conditions & New filters are set and they will affect the next ticket request or visit plan\\
			\hline
			
			Exception & \begin{enumerate}
				\item User closes the app without saving the filters\newline
				Filter modifications are lost
				
			\end{enumerate}\\
			
			\hline
		\end{tabular}
	}
\end{table}

%TABELLA 11
\item \textbf{Edit filters}\\\\

\begin{table}[H]
	{\rowcolors{1}{white}{white}
		\begin{tabular}{|c|p{14cm}|}
			\hline
			Name & Edit filters\\
			\hline
			Actors & User\\
			\hline
			Entry Condition & User pressed filter button from main screen\\
			\hline
			
			Event Flow & \begin{enumerate}
				\item User is in main screen and taps the “filters” button
				\item User is in filters screen
				\item User changes the filter parameters he wants to change among:
				\begin{enumerate}
					\item distance range
					\item store type
					\item default booking time (ignored by ticket request)
					\item default calendar or stores first when booking (ignored by ticket request)
					\item default map or list view when booking (ignored by ticket request)
					\item ???
				\end{enumerate}
				
				\item User presses back to main screen button
				\item Popup to confirm and save or discard the new filters is shown 
				
			\end{enumerate}\\
			
			\hline
			Exit Conditions & New filters are set and they will affect the next ticket request or visit plan\\
			\hline
			
			Exception & \begin{enumerate}
				\item User closes the app without saving the filters\newline
				Filter modifications are lost
				
			\end{enumerate}\\
			
			\hline
		\end{tabular}
	}
\end{table}

%TABELLA 12
\item \textbf{Edit filters}\\\\

\begin{table}[H]
	{\rowcolors{1}{white}{white}
		\begin{tabular}{|c|p{14cm}|}
			\hline
			Name & Edit filters\\
			\hline
			Actors & User\\
			\hline
			Entry Condition & User pressed filter button from main screen\\
			\hline
			
			Event Flow & \begin{enumerate}
				\item User is in main screen and taps the “filters” button
				\item User is in filters screen
				\item User changes the filter parameters he wants to change among:
				\begin{enumerate}
					\item distance range
					\item store type
					\item default booking time (ignored by ticket request)
					\item default calendar or stores first when booking (ignored by ticket request)
					\item default map or list view when booking (ignored by ticket request)
					\item ???
				\end{enumerate}
				
				\item User presses back to main screen button
				\item Popup to confirm and save or discard the new filters is shown 
				
			\end{enumerate}\\
			
			\hline
			Exit Conditions & New filters are set and they will affect the next ticket request or visit plan\\
			\hline
			
			Exception & \begin{enumerate}
				\item User closes the app without saving the filters\newline
				Filter modifications are lost
				
			\end{enumerate}\\
			
			\hline
		\end{tabular}
	}
\end{table}

%TABELLA 13
\item \textbf{Edit filters}\\\\

\begin{table}[H]
	{\rowcolors{1}{white}{white}
		\begin{tabular}{|c|p{14cm}|}
			\hline
			Name & Edit filters\\
			\hline
			Actors & User\\
			\hline
			Entry Condition & User pressed filter button from main screen\\
			\hline
			
			Event Flow & \begin{enumerate}
				\item User is in main screen and taps the “filters” button
				\item User is in filters screen
				\item User changes the filter parameters he wants to change among:
				\begin{enumerate}
					\item distance range
					\item store type
					\item default booking time (ignored by ticket request)
					\item default calendar or stores first when booking (ignored by ticket request)
					\item default map or list view when booking (ignored by ticket request)
					\item ???
				\end{enumerate}
				
				\item User presses back to main screen button
				\item Popup to confirm and save or discard the new filters is shown 
				
			\end{enumerate}\\
			
			\hline
			Exit Conditions & New filters are set and they will affect the next ticket request or visit plan\\
			\hline
			
			Exception & \begin{enumerate}
				\item User closes the app without saving the filters\newline
				Filter modifications are lost
				
			\end{enumerate}\\
			
			\hline
		\end{tabular}
	}
\end{table}

%TABELLA 14
\item \textbf{Edit filters}\\\\

\begin{table}[H]
	{\rowcolors{1}{white}{white}
		\begin{tabular}{|c|p{14cm}|}
			\hline
			Name & Edit filters\\
			\hline
			Actors & User\\
			\hline
			Entry Condition & User pressed filter button from main screen\\
			\hline
			
			Event Flow & \begin{enumerate}
				\item User is in main screen and taps the “filters” button
				\item User is in filters screen
				\item User changes the filter parameters he wants to change among:
				\begin{enumerate}
					\item distance range
					\item store type
					\item default booking time (ignored by ticket request)
					\item default calendar or stores first when booking (ignored by ticket request)
					\item default map or list view when booking (ignored by ticket request)
					\item ???
				\end{enumerate}
				
				\item User presses back to main screen button
				\item Popup to confirm and save or discard the new filters is shown 
				
			\end{enumerate}\\
			
			\hline
			Exit Conditions & New filters are set and they will affect the next ticket request or visit plan\\
			\hline
			
			Exception & \begin{enumerate}
				\item User closes the app without saving the filters\newline
				Filter modifications are lost
				
			\end{enumerate}\\
			
			\hline
		\end{tabular}
	}
\end{table}

%TABELLA 15
\item \textbf{Edit filters}\\\\

\begin{table}[H]
	{\rowcolors{1}{white}{white}
		\begin{tabular}{|c|p{14cm}|}
			\hline
			Name & Edit filters\\
			\hline
			Actors & User\\
			\hline
			Entry Condition & User pressed filter button from main screen\\
			\hline
			
			Event Flow & \begin{enumerate}
				\item User is in main screen and taps the “filters” button
				\item User is in filters screen
				\item User changes the filter parameters he wants to change among:
				\begin{enumerate}
					\item distance range
					\item store type
					\item default booking time (ignored by ticket request)
					\item default calendar or stores first when booking (ignored by ticket request)
					\item default map or list view when booking (ignored by ticket request)
					\item ???
				\end{enumerate}
				
				\item User presses back to main screen button
				\item Popup to confirm and save or discard the new filters is shown 
				
			\end{enumerate}\\
			
			\hline
			Exit Conditions & New filters are set and they will affect the next ticket request or visit plan\\
			\hline
			
			Exception & \begin{enumerate}
				\item User closes the app without saving the filters\newline
				Filter modifications are lost
				
			\end{enumerate}\\
			
			\hline
		\end{tabular}
	}
\end{table}

%TABELLA 16
\item \textbf{Edit filters}\\\\

\begin{table}[H]
	{\rowcolors{1}{white}{white}
		\begin{tabular}{|c|p{14cm}|}
			\hline
			Name & Edit filters\\
			\hline
			Actors & User\\
			\hline
			Entry Condition & User pressed filter button from main screen\\
			\hline
			
			Event Flow & \begin{enumerate}
				\item User is in main screen and taps the “filters” button
				\item User is in filters screen
				\item User changes the filter parameters he wants to change among:
				\begin{enumerate}
					\item distance range
					\item store type
					\item default booking time (ignored by ticket request)
					\item default calendar or stores first when booking (ignored by ticket request)
					\item default map or list view when booking (ignored by ticket request)
					\item ???
				\end{enumerate}
				
				\item User presses back to main screen button
				\item Popup to confirm and save or discard the new filters is shown 
				
			\end{enumerate}\\
			
			\hline
			Exit Conditions & New filters are set and they will affect the next ticket request or visit plan\\
			\hline
			
			Exception & \begin{enumerate}
				\item User closes the app without saving the filters\newline
				Filter modifications are lost
				
			\end{enumerate}\\
			
			\hline
		\end{tabular}
	}
\end{table}

%TABELLA 17
\item \textbf{Edit filters}\\\\

\begin{table}[H]
	{\rowcolors{1}{white}{white}
		\begin{tabular}{|c|p{14cm}|}
			\hline
			Name & Edit filters\\
			\hline
			Actors & User\\
			\hline
			Entry Condition & User pressed filter button from main screen\\
			\hline
			
			Event Flow & \begin{enumerate}
				\item User is in main screen and taps the “filters” button
				\item User is in filters screen
				\item User changes the filter parameters he wants to change among:
				\begin{enumerate}
					\item distance range
					\item store type
					\item default booking time (ignored by ticket request)
					\item default calendar or stores first when booking (ignored by ticket request)
					\item default map or list view when booking (ignored by ticket request)
					\item ???
				\end{enumerate}
				
				\item User presses back to main screen button
				\item Popup to confirm and save or discard the new filters is shown 
				
			\end{enumerate}\\
			
			\hline
			Exit Conditions & New filters are set and they will affect the next ticket request or visit plan\\
			\hline
			
			Exception & \begin{enumerate}
				\item User closes the app without saving the filters\newline
				Filter modifications are lost
				
			\end{enumerate}\\
			
			\hline
		\end{tabular}
	}
\end{table}


\end{enumerate}
\subsection{Performance Requirements}

\subsection{Design Constraints}
Follows a set of constraints on the system design imposed by external standards, regulatory requirements, or project
limitations.

\subsubsection{Standards compliance}
The application does not require adhesion to any standard in particular. 
\subsubsection{Hardware limitations}

\subsubsection{Any other constraint}


\subsection{Software System Attributes}

\subsubsection{Reliability}

\subsubsection{Availability}

\subsubsection{Security}

\subsubsection{Maintainability}

\subsubsection{Portability}
