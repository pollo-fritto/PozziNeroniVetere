Organize this section according to the rules defined in the project description. \\ 

\subsection{\sffamily External Interface Requirements}

\subsubsection{\sffamily User Interfaces}
Here will go wireframes
\subsubsection{\sffamily Hardware Interfaces}
Both users and store managers can use the application through a mobile phone or a personal computer. Users unable to do so will use totems provided by stores.
\subsubsection{\sffamily Software Interfaces}
map API\\
send ticket/booking requests to CLup\\
send turnstile entrances/exits to CLup\\
query automatically generated reports and obtain their statistics\\
\subsubsection{\sffamily Communication Interfaces}
The only type of communication required by CLup is a stable internet connection.
\subsection{\sffamily Functional Requirements}

\subsubsection{\sffamily List of requirements}
\begin{table}[H]
\label{tab: ReqList}
\begin{tabular}{l|l}
	R1 & Every user can generate a quick ticket for any store \\
	R2 & \pbox{13cm}{Whenever user makes initiates a booking procedure, CLup must be able to compute a suggested least crowded time slot based on historical data} \\
	R3 & \pbox{13cm}{CLup must elaborate and upload data about current global customer affluence to the store during use} \\
	R4 & CLup must admit only valid QR codes for entrance \\
	R5 & CLup must allow users to know current queue status \\
	R6 & CLup must update user on tickets' validity change \\
	R7 & CLup must inform offline users about new tickets (un)availability \\
	R8 & CLup must allow users to indicate which product category they are going to purchase while booking \\
	R9 & CLup must suggest alternative stores when the combination of selected store|time gives no results \\
	R10 & CLup must reserve a non null number of paper tickets at any time for offline customers use 	\\
	R11 & CLup must gather all stores' data about entrance fluxes \\
	R12 & CLup is able to cross affluence data of any supermarket\\
	R13 & CLup keeps track of people who book an entrance and don’t come\\
	R14 & CLup allows store managers to stop quick tickets availability \\
	R15 & CLup is able to generate QR codes\\
	R16 & CLup is able to authenticate users\\
	R17 & CLup is able to store users’ data \\
	R18 & CLup is able to process users' data \\
	R19 & CLup makes quick ticket invalid after 15 minutes delay\\
	R20 & CLup can use stores’ data to sort every store by crowdedness \\
	R21 & Users can see available day/time slots of a supermarket through CLup\\
	R22 & CLup shows to store managers flux data about their supermarket (forse questo è quello che intendeva R3)\\
	R23 & CLup must be able to process reservations\\
	
\end{tabular}
\caption{Requirements list}
\end{table}

\subsubsection{\sffamily Mapping}
\begin{table} [H]
	\begin{tabular}{c|c|c}
		\textbf{Goals} & \textbf{Requirements} & \textbf{Domain Assumptions}\\
		\hline
		\textbf{G1} & R1, R6, R7, R10, R15,R23 & D1,D5,D8\\
		\hline
		\textbf{G2} & R2, R11, R12, R18, R21 & D1, D2, D8, D10\\
		\hline
		\textbf{G3} & R9, R13, R15, R21, R23 & D5, D8, D10\\
		\hline
		\textbf{G4} & R3, R11, R22 & D1, D2, D8, D11\\
		\hline
		\textbf{G5} & R4, R6, R13, R15, R19 & D1, D2, D3, D8, D10, D11\\
		\hline
		\textbf{G6} & R1, R2, R5, R6 & D4, D5, D6, D7, D8, D10\\
		\hline
		\textbf{G7} & R1, R5, R6, R7, R9, R10, R20, R21, R23 & D2, D4, D5, D8, D9, D10\\
		\hline
		\textbf{G8} & R1, R7, R10 & D6, D7, D8, D10\\
		\hline
		\textbf{G9} & R3, R11, R12, R20 & D1, D2, D8, D9, D10, D11\\
		\hline
		\textbf{G10} & R2, R3, R4, R11, R12, R20 & D1, D2, D3, D7, D8, D10, D11\\
	\end{tabular}
	\caption{Goal mapping summary}
	\label{tab:MappingSum}
\end{table}

\begin{table}[H]
	\rowcolors{1}{white}{white}
	\begin{tabular}{c|l}
		\cellcolor{lightgray}\textbf{G1} & \pbox{13cm}{\textbf{Anybody is guaranteed possibility to make shopping at any supermarket in reasonable time (def. reasonable)}}\\
		\hline
		\cellcolor{YellowGreen} R1 & Every user can generate a quick ticket for any store\\
		\hline
		\cellcolor{YellowGreen} R6 & CLup must update user on tickets' validity change\\
		\hline
		\cellcolor{YellowGreen} R7 & CLup must inform offline users about new tickets (un)availability \\
		\hline
		\cellcolor{YellowGreen} R10 & CLup must reserve a non null number of paper tickets at any time for offline customers use\\
		\hline
		\cellcolor{YellowGreen} R15 & CLup is able to generate QR codes\\
		\hline
		\cellcolor{YellowGreen} R23 & CLup must be able to process reservations\\
		\hline
		\cellcolor{YellowOrange} D1 & Accesses to the store can be monitored\\
		\hline
		\cellcolor{YellowOrange} D5 & Users can estimate the time required to arrive to the store\\
		\hline
		\cellcolor{YellowOrange} D8 & Malicious users are not enough in number or coordination to prevent Clup to work\\
	\end{tabular}
	\label{tab:G1Mapping}
	\caption{G1 Mapping}
\end{table}

\begin{table}[H]
	\rowcolors{1}{white}{white}
	\begin{tabular}{c|l}
		\cellcolor{lightgray}\textbf{G2} & \textbf{Users can get to know the least crowded time slots}\\
		\hline
		\cellcolor{YellowGreen} R2 & \pbox{13cm}{Whenever user makes initiates a booking procedure, CLup must be able to compute a suggested least crowded time slot based on historical data}\\
		\hline
		\cellcolor{YellowGreen} R11 & CLup must gather all stores' data about entrance fluxes\\
		\hline
		\cellcolor{YellowGreen} R12 & CLup is able to cross affluence data of any supermarket\\
		\hline
		\cellcolor{YellowGreen} R18 & CLup is able to process users' data \\
		\hline
		\cellcolor{YellowGreen} R21 & Users can see available day/time slots of a supermarket through CLup\\
		\hline
		\cellcolor{YellowOrange} D1 & Accesses to the store can be monitored\\
		\hline
		\cellcolor{YellowOrange} D2 & Exits from the store can be monitored\\
		\hline
		\cellcolor{YellowOrange} D8 & Malicious users are not enough in number or coordination to prevent Clup to work\\
		\hline
		\cellcolor{YellowOrange} D10 & Store managers give the right information about supermarkets\\
	\end{tabular}
	\label{tab:G2Mapping}
	\caption{G2 Mapping}
\end{table}

\begin{table}[H]
	\rowcolors{1}{white}{white}
	\begin{tabular}{c|l}
		\cellcolor{lightgray}\textbf{G3} & \textbf{Fair users can make a reservation to enter in a supermarket}\\
		\hline
		\cellcolor{YellowGreen} R9 & CLup must suggest alternative stores when the combination of selected store|time gives no results\\
		\hline
		\cellcolor{YellowGreen} R13 & CLup keeps track of people who book an entrance and don’t come\\
		\hline
		\cellcolor{YellowGreen} R15 & CLup is able to generate QR codes \\
		\hline
		\cellcolor{YellowGreen} R21 & Users can see available day/time slots of a supermarket through CLup\\
		\hline
		\cellcolor{YellowGreen} R23 & CLup must be able to process reservations\\
		\hline
		\cellcolor{YellowOrange} D5 & Users can estimate the time required to arrive to the store\\
		\hline
		\cellcolor{YellowOrange} D8 & Malicious users are not enough in number or coordination to prevent Clup to work\\
		\hline
		\cellcolor{YellowOrange} D10 & Store managers give the right information about supermarkets\\
	\end{tabular}
	\label{tab:G3Mapping}
	\caption{G3 Mapping}
\end{table}

\begin{table}[H]
	\rowcolors{1}{white}{white}
	\begin{tabular}{c|l}
		\cellcolor{lightgray}\textbf{G4} & \textbf{Stores can easily monitor fluxes}\\
		\hline
		\cellcolor{YellowGreen} R3 & CLup must elaborate and upload data about current global customer affluence to the store during use\\
		\hline
		\cellcolor{YellowGreen} R11 & CLup must gather all stores' data about entrance fluxes\\
		\hline
		\cellcolor{YellowGreen} R22 & CLup shows to store managers flux data about their supermarket \\
		\hline
		\cellcolor{YellowOrange} D1 & Accesses to the store can be monitored\\
		\hline
		\cellcolor{YellowOrange} D2 & Exits from the store can be monitored\\
		\hline
		\cellcolor{YellowOrange} D8 & Malicious users are not enough in number or coordination to prevent Clup to work\\
		\hline
		\cellcolor{YellowOrange} D11 & Staff guarantees access control systems operativeness\\
	\end{tabular}
	\label{tab:G4Mapping}
	\caption{G4 Mapping}
\end{table}

\begin{table}[H]
	\rowcolors{1}{white}{white}
	\begin{tabular}{c|l}
		\cellcolor{lightgray}\textbf{G5} & \textbf{Only authorized users can access}\\
		\hline
		\cellcolor{YellowGreen} R4 & CLup must admit only valid QR codes for entrance \\
		\hline
		\cellcolor{YellowGreen} R6 & CLup must update user on tickets' validity change\\
		\hline
		\cellcolor{YellowGreen} R13 & CLup keeps track of people who book an entrance and don’t come\\
		\hline
		\cellcolor{YellowGreen} R15 & CLup is able to generate QR codes\\
		\hline
		\cellcolor{YellowGreen} R19 & CLup makes quick ticket invalid after 15 minutes delay\\
		\hline
		\cellcolor{YellowOrange} D1 & Accesses to the store can be monitored\\
		\hline
		\cellcolor{YellowOrange} D2 & Exits from the store can be monitored\\
		\hline
		\cellcolor{YellowOrange} D3 & One customer per authorization given is allowed in by the Staff\\
		\hline
		\cellcolor{YellowOrange} D8 & Malicious users are not enough in number or coordination to prevent Clup to work\\
		\hline
		\cellcolor{YellowOrange} D10 & Store managers give the right information about supermarkets\\
		\hline
		\cellcolor{YellowOrange} D11 & Staff guarantees access control systems operativeness\\
	\end{tabular}
	\label{tab:G5Mapping}
	\caption{G5 Mapping}
\end{table}

\begin{table}[H]
	\rowcolors{1}{white}{white}
	\begin{tabular}{c|l}
		\cellcolor{lightgray}\textbf{G6} & \textbf{Crowds are dramatically reduced outside supermarket stores}\\
		\hline
		\cellcolor{YellowGreen} R1 & Every user can generate a quick ticket for any store\\
		\hline
		\cellcolor{YellowGreen} R2 & Whenever user makes initiates a booking procedure, CLup must be able to compute a suggested least crowded time slot based on historical data  \\
		\hline
		\cellcolor{YellowGreen} R5 & CLup must allow users to know current queue status\\
		\hline
		\cellcolor{YellowGreen} R6 & CLup must update user on tickets' validity change\\
		\hline
		\cellcolor{YellowOrange} D4 & Users are reasonably able to manage their time while following the queue evolution\\
		\hline
		\cellcolor{YellowOrange} D5 & Users can estimate the time required to arrive to the store\\
		\hline
		\cellcolor{YellowOrange} D6 & Users who arrives too early at the supermarket don't wait in front of the entrance\\
		\hline
		\cellcolor{YellowOrange} D7 & Customers keep the safe distance\\
		\hline
		\cellcolor{YellowOrange} D8 & Malicious users are not enough in number or coordination to prevent Clup to work\\
		\hline
		\cellcolor{YellowOrange} D10 & Store managers give the right information about supermarkets\\
	\end{tabular}
	\label{tab:G6Mapping}
	\caption{G6 Mapping}
\end{table}

\begin{table}[H]
	\rowcolors{1}{white}{white}
	\begin{tabular}{c|l}
		\cellcolor{lightgray}\textbf{G7} & \pbox{13cm}{\textbf{CLup should not decrease customer affluence beyond a reasonable level w.r.t. to normal (→ define reasonable)}}\\
		\hline
		\cellcolor{YellowGreen} R1 & Every user can generate a quick ticket for any store\\
		\hline
		\cellcolor{YellowGreen} R5 & CLup must allow users to know current queue status\\
		\hline
		\cellcolor{YellowGreen} R6 & CLup must update user on tickets' validity change\\
		\hline
		\cellcolor{YellowGreen} R7 & CLup must inform offline users about new tickets (un)availability \\
		\hline
		\cellcolor{YellowGreen} R9 & CLup must suggest alternative stores when the combination of selected store|time gives no results \\
		\hline
		\cellcolor{YellowGreen} R10 & CLup must reserve a non null number of paper tickets at any time for offline customers use\\
		\hline
		\cellcolor{YellowGreen} R20 & CLup can use stores’ data to sort every store by crowdedness\\
		\hline
		\cellcolor{YellowGreen} R21 & Users can see available day/time slots of a supermarket through CLup\\
		\hline
		\cellcolor{YellowGreen} R23 & CLup must be able to process reservations\\
		\hline
		\cellcolor{YellowOrange} D2 & Exits from the store can be monitored\\
		\hline
		\cellcolor{YellowOrange} D4 & Users are reasonably able to manage their time while following the queue evolution\\
		\hline
		\cellcolor{YellowOrange} D5 & Users can estimate the time required to arrive to the store\\
		\hline
		\cellcolor{YellowOrange} D8 & Malicious users are not enough in number or coordination to prevent Clup to work\\
		\cellcolor{YellowOrange} D9 & Users insert the right starting location or their GPS works\\
		\hline
		\cellcolor{YellowOrange} D10 & Store managers give the right information about supermarkets\\
		\hline
	\end{tabular}
	\label{tab:G7Mapping}
	\caption{G7 Mapping}
\end{table}

\begin{table}[H]
	\rowcolors{1}{white}{white}
	\begin{tabular}{c|l}
		\cellcolor{lightgray}\textbf{G8} & \textbf{Same shopping capabilities guaranteed to offline users}\\
		\hline
		\cellcolor{YellowGreen} R1 & Every user can generate a quick ticket for any store\\
		\hline
		\cellcolor{YellowGreen} R7 & CLup must inform offline users about new tickets (un)availability \\
		\hline
		\cellcolor{YellowGreen} R10 & CLup must reserve a non null number of paper tickets at any time for offline customers use\\
		\hline
		\cellcolor{YellowOrange} D6 & Users who arrives too early at the supermarket don't wait in front of the entrance\\
		\hline
		\cellcolor{YellowOrange} D7 & Customers keep the safe distance\\
		\hline
		\cellcolor{YellowOrange} D8 & Malicious users are not enough in number or coordination to prevent Clup to work\\
		\hline
		\cellcolor{YellowOrange} D7 & Customers keep the safe distance\\
	\end{tabular}
	\label{tab:G8Mapping}
	\caption{G8 Mapping}
\end{table}

\begin{table}[H]
	\rowcolors{1}{white}{white}
	\begin{tabular}{c|l}
		\cellcolor{lightgray}\textbf{G9} & \pbox{13cm}{\textbf{Find the best (less crowded, soonest available) alternative among local supermarket stores (of same franchise only?)}}\\
		\hline
		\cellcolor{YellowGreen} R3 & CLup must elaborate and upload data about current global customer affluence to the store during use \\
		\hline
		\cellcolor{YellowGreen} R11 & CLup must gather all stores' data about entrance fluxes \\
		\hline
		\cellcolor{YellowGreen} R12 & CLup is able to cross affluence data of any supermarket\\
		\hline
		\cellcolor{YellowGreen} R20 & CLup can use stores’ data to sort every store by crowdedness\\
		\hline
		\cellcolor{YellowOrange} D1 & Accesses to the store can be monitored\\
		\hline
		\cellcolor{YellowOrange} D2 & Exits from the store can be monitored\\
		\hline
		\cellcolor{YellowOrange} D8 & Malicious users are not enough in number or coordination to prevent Clup to work\\
		\hline
		\cellcolor{YellowOrange} D9 & Users insert the right starting location or their GPS works\\
		\hline
		\cellcolor{YellowOrange} D10 & Store managers give the right information about supermarkets\\
		\hline
		\cellcolor{YellowOrange} D11 & Staff guarantees access control systems operativeness\\
	\end{tabular}
	\label{tab:G9Mapping}
	\caption{G9 Mapping}
\end{table}

\begin{table}[H]
	\rowcolors{1}{white}{white}
	\begin{tabular}{c|l}
		\cellcolor{lightgray}\textbf{G10} & \pbox{13cm}{\textbf{Supermarkets do not overcrowd}}\\
		\hline
		\cellcolor{YellowGreen} R2 & \pbox{13cm}{Whenever user makes initiates a booking procedure, CLup must be able to compute a suggested least crowded time slot based on historical data} \\
		\hline
		\cellcolor{YellowGreen} R3 & CLup must elaborate and upload data about current global customer affluence to the store during use\\
		\hline
		\cellcolor{YellowGreen} R4 & CLup must admit only valid QR codes for entrance\\
		\hline
		\cellcolor{YellowGreen} R11 & CLup must gather all stores' data about entrance fluxes\\
		\hline
		\cellcolor{YellowGreen} R12 & CLup is able to cross affluence data of any supermarket\\
		\hline
		\cellcolor{YellowGreen} R20 & CLup can use stores’ data to sort every store by crowdedness\\
		\hline
		\cellcolor{YellowOrange} D1 & Accesses to the store can be monitored\\
		\hline
		\cellcolor{YellowOrange} D2 & Exits from the store can be monitored\\
		\hline
		\cellcolor{YellowOrange} D3 & One customer per authorization given is allowed in by the Staff\\
		\hline
		\cellcolor{YellowOrange} D7 & Customers keep the safe distance\\
		\hline
		\cellcolor{YellowOrange} D8 & Malicious users are not enough in number or coordination to prevent Clup to work\\
		\hline
		\cellcolor{YellowOrange} D10 & Store managers give the right information about supermarkets\\
		\hline
		\cellcolor{YellowOrange} D11 & Staff guarantees access control systems operativeness\\
	\end{tabular}
	\label{tab:G10Mapping}
	\caption{G10 Mapping}
\end{table}

\subsubsection{\sffamily Use cases}

\begin{enumerate}
	
%TABELLA 1
\item \textbf{Registration of new account}

\begin{table}[H]
{\rowcolors{1}{white}{white}
\begin{tabular}{|c|p{14cm}|}
	\hline
	Name & Registration of new account\\
	\hline
	Actors & User\\
	\hline
	Entry Condition & User installed and opened the app and doesn’t have an account or wants to register another one\\
	\hline
	Event Flow & \begin{enumerate}
		\item User opens the app.
		\item Login screen loads.
		\item User taps “Sign up".
		\item TODO REGISTRATION WIREFRAME.
		\end{enumerate}\\
	\hline
	Exit Conditions & User now has an account with which he can log in\\
	\hline
	Exception & \begin{enumerate}
		\item There is no internet when the user presses “create account”
	\end{enumerate}
	
	“No internet” popup, the user can either wait for internet to come back or discard the incomplete account creation by going back to main screen\\
	\hline
\end{tabular}
}
\end{table}
%TABELLA 2
\item \textbf{User login}
	
\begin{table}[H]
{\rowcolors{1}{white}{white}
	\begin{tabular}{|c|p{14cm}|}
		\hline
		Name & User login\\
		\hline
		Actors & User\\
		\hline
		Entry Condition & User already has an account\\
		\hline
		Event Flow & \begin{enumerate}
			\item User opens the app.
			\item Login screen loads.
			\item User already has an account.
			\item User inputs his credentials and presses “login”
			\item Main screen loads
		\end{enumerate}\\
		\hline
		Exit Conditions & User logged in and is now in main screen, from where he can virtually access all of the app’s functionalities\\
		\hline
		Exception & \begin{enumerate}
			\item Credentials are wrong\newline
			wrong credentials popup, user stays in  login screen
			
			\item There is no internet connection\newline
			no internet warning pop up, user still logs in if he used previously inserted and saved credentials so he can still edit settings and filters or look at his history 
			
		\end{enumerate}\\
		
		\hline
	\end{tabular}
}
\end{table}

%TABELLA 3
\item \textbf{Quick ticket request}
	
\begin{table}[H]
{\rowcolors{1}{white}{white}
	\begin{tabular}{|c|p{14cm}|}
		\hline
		Name & ASAP ticket request\\
		\hline
		Actors & User\\
		\hline
		Entry Condition & User successfully logged in and is in main screen\\
		\hline
		
		Event Flow & \begin{enumerate}
			\item User taps “Show stores”
			\item User is taken to “Quick results” screen
			\item IF LIST MODE (from settings)\newline
			stores with queue time, size and distance are shown in a list according to setted filters, if show more is pressed more stores are loaded and the list is made scrollable, if location button is pressed a map showing the store’s location is shown (TODO browse stores on map)\newline
			IF MAP MODE (from settings\newline)
			stores with queue time, size and distance are shown on a map, if list button is pressed User is taken to the previously descripted “list mode” case
			\item User taps on a store
			\item Confirmation pop up is shown
			\item If user confirms “Ticket screen” is shown, otherwise he is taken back to 3
		\end{enumerate}\\
	
		\hline
		Exit Conditions & User has a ticket with updating due time to enter the store\\
		\hline
		
		Exception & \begin{enumerate}
			\item The store blocked ticket requests\newline
			after 5 user is told that the store is no longer available and remains in list/map to make an eventual different choice
			
			\item There is no internet connection\newline
			after 1 user stays in main screen with a dismissable “no internet” pop up
			
			\item User already has an ASAP ticket\newline
			after 1 user stays in main screen with a dismissable “you can only have one ASAP ticket” 
			
		\end{enumerate}\\
		
		\hline
	\end{tabular}
}
\end{table}

%TABELLA 4
\item \textbf{Quick ticket request at physical Totem}

\begin{table}[H]
{\rowcolors{1}{white}{white}
	\begin{tabular}{|c|p{14cm}|}
		\hline
		Name & ASAP ticket request at physical Totem\\
		\hline
		Actors & User\\
		\hline
		Entry Condition & User starts interacting with CLup tablet (?) outside the store\\
		\hline
		
		Event Flow & \begin{enumerate}
			\item POSSIBLE IDENTIFICATION (fiscal code or ID card)
			\item User sees current queue for this store and decides whether to confirm or cancel
			\item A card ticket with remainder and QR is printed
			\item User leaves the station with his printed ticket
			
		\end{enumerate}\\
		
		\hline
		Exit Conditions & User has a ticket that he can scan to enter the store when he comes back at the written date and time\\
		\hline
		
		Exception & \begin{enumerate}
			\item The store blocked ticket requests\newline
			User can’t get a ticket and is asked to come back another time
			
		\end{enumerate}\\
		
		\hline
	\end{tabular}
}
\end{table}

%TABELLA 5
\item \textbf{Edit filters}

\begin{table}[H]
{\rowcolors{1}{white}{white}
	\begin{tabular}{|c|p{14cm}|}
		\hline
		Name & Edit filters\\
		\hline
		Actors & User\\
		\hline
		Entry Condition & User pressed filter button from main screen\\
		\hline
		
		Event Flow & \begin{enumerate}
			\item User is in main screen and taps the “filters” button
			\item User is in filters screen
			\item User changes the filter parameters he wants to change among:
			\begin{enumerate}
				\item distance range
				\item store type
				\item default booking time (ignored by ticket request)
				\item default calendar or stores first when booking (ignored by ticket request)
				\item default map or list view when booking (ignored by ticket request)
				\item ???
			\end{enumerate}
			
			\item User presses back to main screen button
			\item Popup to confirm and save or discard the new filters is shown 

		\end{enumerate}\\
		
		\hline
		Exit Conditions & New filters are set and they will affect the next ticket request or visit plan\\
		\hline
	
		Exception & \begin{enumerate}
			\item User closes the app without saving the filters\newline
			Filter modifications are lost
			
		\end{enumerate}\\
		
		\hline
	\end{tabular}
}
\end{table}

%TABELLA 6 
\item \textbf{Plan visit}

\begin{table}[H]
{\rowcolors{1}{white}{white}
	\begin{tabular}{|c|p{14cm}|}
		\hline
		Name & ASAP ticket request\\
		\hline
		Actors & User\\
		\hline
		Entry Condition & User successfully logged in and is in main screen\\
		\hline
		
		Event Flow & \begin{enumerate}
			\item User taps “Show stores”
			\item User is taken to “Quick results”
			\item User inputs the expected amount of time he will be shopping
			\item Calendar or store map/list is shown depending on what was loaded given the filter settings
			\item User chooses date or store (from map/list) 
			\item Stores map/list or calendar depending on what was shown already
			\item User chooses the store or the day
			\item CLup shows time slots for that store on that day
			\item User taps on a time slot
			\item Confirmation pop up is shown
			\item If user confirms “Ticket screen” is shown, otherwise he is taken back to 3			
		\end{enumerate}\\
		
		\hline
		Exit Conditions & User has a ticket with updating due time to enter the store\\
		\hline
		
		Exception & \begin{enumerate}
			\item The store blocked ticket requests\newline
			after 5 user is told that the store is no longer available and remains in list/map to make an eventual different choice
			
			\item There is no internet connection\newline
			after 1 user stays in main screen with a dismissable “no internet” pop up
			
			\item User already has an ASAP ticket\newline
			after 1 user stays in main screen with a dismissable “you can only have one ASAP ticket”
			
			\item Day/store/time slot is full\newline
			the choice is refused with a popup and user has to choose an alternative
			
		\end{enumerate}\\
		
		\hline
	\end{tabular}
}
\end{table}

%TABELLA 7 TODO da copiare/incollare i vari campi!
\item \textbf{Enter store}

\begin{table}[H]
	{\rowcolors{1}{white}{white}
		\begin{tabular}{|c|p{14cm}|}
			\hline
			Name & Enter store\\
			\hline
			Actors & User\\
			\hline
			Entry Condition & User has a valid QR ticket and is at the store entrance\\
			\hline
			
			Event Flow & \begin{enumerate}
				\item User scans his QR ticket or inputs his alphanumeric code at the turnstile
				\item The turnstile informs the user he can enter and unlocks
				\item The user enters the store
				
			\end{enumerate}\\
			
			\hline
			Exit Conditions & The user is in the store and can start shopping\\
			\hline
			
			Exception & \begin{enumerate}
				\item The ticket expired its 15 minutes validity time\newline
				The turnstile remains locked and informs the user that his ticket is not valid, the ticket expiration causes a report to be generated for the manager independently from the fact that the user tried to enter anyway
				
				\item The code is wrong\newline
				The turnstile remains locked and informs the user that the code is not valid

			\end{enumerate}\\
			
			\hline
		\end{tabular}
	}
\end{table}

%TABELLA 8
\item \textbf{Exit store}

\begin{table}[H]
	{\rowcolors{1}{white}{white}
		\begin{tabular}{|c|p{14cm}|}
			\hline
			Name & Exit store\\
			\hline
			Actors & User\\
			\hline
			Entry Condition & User has a valid QR ticket and is exiting the store\\
			\hline
			
			Event Flow & \begin{enumerate}
				\item User scans his QR ticket or inputs his alphanumeric code or inputs his CLup email or nickname at the turnstile
				\item The turnstile informs the user he can exit and unlocks
				\item The user exits the store
				
			\end{enumerate}\\
			
			\hline
			Exit Conditions & The user is in the store and can start shopping\\
			\hline
			
			Exception & \begin{enumerate}
				\item The code is wrong or no code can be provided or credentials are wrong\newline
				The turnstile informs the user that he has to try again and after the third try it is unlocked anyway, the user exit won’t be tracked and will cause a report for the misinterpreted long shopping to be generated for him anyway
				
				
			\end{enumerate}\\
			
			\hline
		\end{tabular}
	}
\end{table}

%TABELLA 9
\item \textbf{Store staff stops new entrances}

\begin{table}[H]
	{\rowcolors{1}{white}{white}
		\begin{tabular}{|c|p{14cm}|}
			\hline
			Name & Stops new entrances\\
			\hline
			Actors & Store manager\\
			\hline
			Entry Condition & Store manager is in management web page\\
			\hline
			
			Event Flow & \begin{enumerate}
				\item Staff clicks on stop new entrances
				\item Staff is asked to insert admin password
				\item Staff is asked to confirm and he does
				
			\end{enumerate}\\
			
			\hline
			Exit Conditions & No new tickets can be issued and currently released tickets are put on hold, meaning that valid tickets won’t be accepted by the turnstiles (queue time replaced with a suspened entrances message)
			The page now offers the possibility to allow entrances again\\
			\hline
			
			Exception & None\\
			
			\hline
		\end{tabular}
	}
\end{table}

%TABELLA 10
\item \textbf{Store manager view affluence statistics}

\begin{table}[H]
	{\rowcolors{1}{white}{white}
		\begin{tabular}{|c|p{14cm}|}
			\hline
			Name & View affluence statistics\\
			\hline
			Actors & Store manager\\
			\hline
			Entry Condition & The manager is in management web page\\
			\hline
			
			Event Flow & \begin{enumerate}
				\item Staff clicks on “View affluence statistics”
				\item A page showing various statistics with customizable filters and options about people affluences is shown
				
			\end{enumerate}\\
			
			\hline
			Exit Conditions & Manager knows their customers’ behaviours (e.g. affluence and permanence times) and can act accordingly\\
			\hline
			
			Exception & \begin{enumerate}
				\item None
				
			\end{enumerate}\\
			
			\hline
		\end{tabular}
	}
\end{table}

%TABELLA 11
\item \textbf{Edit filters}

\begin{table}[H]
	{\rowcolors{1}{white}{white}
		\begin{tabular}{|c|p{14cm}|}
			\hline
			Name & Edit filters\\
			\hline
			Actors & User\\
			\hline
			Entry Condition & User pressed filter button from main screen\\
			\hline
			
			Event Flow & \begin{enumerate}
				\item User is in main screen and taps the “filters” button
				\item User is in filters screen
				\item User changes the filter parameters he wants to change among:
				\begin{enumerate}
					\item distance range
					\item store type
					\item default booking time (ignored by ticket request)
					\item default calendar or stores first when booking (ignored by ticket request)
					\item default map or list view when booking (ignored by ticket request)
					\item ???
				\end{enumerate}
				
				\item User presses back to main screen button
				\item Popup to confirm and save or discard the new filters is shown 
				
			\end{enumerate}\\
			
			\hline
			Exit Conditions & New filters are set and they will affect the next ticket request or visit plan\\
			\hline
			
			Exception & \begin{enumerate}
				\item User closes the app without saving the filters\newline
				Filter modifications are lost
				
			\end{enumerate}\\
			
			\hline
		\end{tabular}
	}
\end{table}

%TABELLA 12
\item \textbf{Edit filters}

\begin{table}[H]
	{\rowcolors{1}{white}{white}
		\begin{tabular}{|c|p{14cm}|}
			\hline
			Name & Edit filters\\
			\hline
			Actors & User\\
			\hline
			Entry Condition & User pressed filter button from main screen\\
			\hline
			
			Event Flow & \begin{enumerate}
				\item User is in main screen and taps the “filters” button
				\item User is in filters screen
				\item User changes the filter parameters he wants to change among:
				\begin{enumerate}
					\item distance range
					\item store type
					\item default booking time (ignored by ticket request)
					\item default calendar or stores first when booking (ignored by ticket request)
					\item default map or list view when booking (ignored by ticket request)
					\item ???
				\end{enumerate}
				
				\item User presses back to main screen button
				\item Popup to confirm and save or discard the new filters is shown 
				
			\end{enumerate}\\
			
			\hline
			Exit Conditions & New filters are set and they will affect the next ticket request or visit plan\\
			\hline
			
			Exception & \begin{enumerate}
				\item User closes the app without saving the filters\newline
				Filter modifications are lost
				
			\end{enumerate}\\
			
			\hline
		\end{tabular}
	}
\end{table}

%TABELLA 13
\item \textbf{Edit filters}

\begin{table}[H]
	{\rowcolors{1}{white}{white}
		\begin{tabular}{|c|p{14cm}|}
			\hline
			Name & Edit filters\\
			\hline
			Actors & User\\
			\hline
			Entry Condition & User pressed filter button from main screen\\
			\hline
			
			Event Flow & \begin{enumerate}
				\item User is in main screen and taps the “filters” button
				\item User is in filters screen
				\item User changes the filter parameters he wants to change among:
				\begin{enumerate}
					\item distance range
					\item store type
					\item default booking time (ignored by ticket request)
					\item default calendar or stores first when booking (ignored by ticket request)
					\item default map or list view when booking (ignored by ticket request)
					\item ???
				\end{enumerate}
				
				\item User presses back to main screen button
				\item Popup to confirm and save or discard the new filters is shown 
				
			\end{enumerate}\\
			
			\hline
			Exit Conditions & New filters are set and they will affect the next ticket request or visit plan\\
			\hline
			
			Exception & \begin{enumerate}
				\item User closes the app without saving the filters\newline
				Filter modifications are lost
				
			\end{enumerate}\\
			
			\hline
		\end{tabular}
	}
\end{table}

%TABELLA 14
\item \textbf{Edit filters}

\begin{table}[H]
	{\rowcolors{1}{white}{white}
		\begin{tabular}{|c|p{14cm}|}
			\hline
			Name & Edit filters\\
			\hline
			Actors & User\\
			\hline
			Entry Condition & User pressed filter button from main screen\\
			\hline
			
			Event Flow & \begin{enumerate}
				\item User is in main screen and taps the “filters” button
				\item User is in filters screen
				\item User changes the filter parameters he wants to change among:
				\begin{enumerate}
					\item distance range
					\item store type
					\item default booking time (ignored by ticket request)
					\item default calendar or stores first when booking (ignored by ticket request)
					\item default map or list view when booking (ignored by ticket request)
					\item ???
				\end{enumerate}
				
				\item User presses back to main screen button
				\item Popup to confirm and save or discard the new filters is shown 
				
			\end{enumerate}\\
			
			\hline
			Exit Conditions & New filters are set and they will affect the next ticket request or visit plan\\
			\hline
			
			Exception & \begin{enumerate}
				\item User closes the app without saving the filters\newline
				Filter modifications are lost
				
			\end{enumerate}\\
			
			\hline
		\end{tabular}
	}
\end{table}

%TABELLA 15
\item \textbf{Edit filters}

\begin{table}[H]
	{\rowcolors{1}{white}{white}
		\begin{tabular}{|c|p{14cm}|}
			\hline
			Name & Edit filters\\
			\hline
			Actors & User\\
			\hline
			Entry Condition & User pressed filter button from main screen\\
			\hline
			
			Event Flow & \begin{enumerate}
				\item User is in main screen and taps the “filters” button
				\item User is in filters screen
				\item User changes the filter parameters he wants to change among:
				\begin{enumerate}
					\item distance range
					\item store type
					\item default booking time (ignored by ticket request)
					\item default calendar or stores first when booking (ignored by ticket request)
					\item default map or list view when booking (ignored by ticket request)
					\item ???
				\end{enumerate}
				
				\item User presses back to main screen button
				\item Popup to confirm and save or discard the new filters is shown 
				
			\end{enumerate}\\
			
			\hline
			Exit Conditions & New filters are set and they will affect the next ticket request or visit plan\\
			\hline
			
			Exception & \begin{enumerate}
				\item User closes the app without saving the filters\newline
				Filter modifications are lost
				
			\end{enumerate}\\
			
			\hline
		\end{tabular}
	}
\end{table}

%TABELLA 16
\item \textbf{Edit filters}

\begin{table}[H]
	{\rowcolors{1}{white}{white}
		\begin{tabular}{|c|p{14cm}|}
			\hline
			Name & Edit filters\\
			\hline
			Actors & User\\
			\hline
			Entry Condition & User pressed filter button from main screen\\
			\hline
			
			Event Flow & \begin{enumerate}
				\item User is in main screen and taps the “filters” button
				\item User is in filters screen
				\item User changes the filter parameters he wants to change among:
				\begin{enumerate}
					\item distance range
					\item store type
					\item default booking time (ignored by ticket request)
					\item default calendar or stores first when booking (ignored by ticket request)
					\item default map or list view when booking (ignored by ticket request)
					\item ???
				\end{enumerate}
				
				\item User presses back to main screen button
				\item Popup to confirm and save or discard the new filters is shown 
				
			\end{enumerate}\\
			
			\hline
			Exit Conditions & New filters are set and they will affect the next ticket request or visit plan\\
			\hline
			
			Exception & \begin{enumerate}
				\item User closes the app without saving the filters\newline
				Filter modifications are lost
				
			\end{enumerate}\\
			
			\hline
		\end{tabular}
	}
\end{table}

%TABELLA 17
\item \textbf{Edit filters}

\begin{table}[H]
	{\rowcolors{1}{white}{white}
		\begin{tabular}{|c|p{14cm}|}
			\hline
			Name & Edit filters\\
			\hline
			Actors & User\\
			\hline
			Entry Condition & User pressed filter button from main screen\\
			\hline
			
			Event Flow & \begin{enumerate}
				\item User is in main screen and taps the “filters” button
				\item User is in filters screen
				\item User changes the filter parameters he wants to change among:
				\begin{enumerate}
					\item distance range
					\item store type
					\item default booking time (ignored by ticket request)
					\item default calendar or stores first when booking (ignored by ticket request)
					\item default map or list view when booking (ignored by ticket request)
					\item ???
				\end{enumerate}
				
				\item User presses back to main screen button
				\item Popup to confirm and save or discard the new filters is shown 
				
			\end{enumerate}\\
			
			\hline
			Exit Conditions & New filters are set and they will affect the next ticket request or visit plan\\
			\hline
			
			Exception & \begin{enumerate}
				\item User closes the app without saving the filters\newline
				Filter modifications are lost
				
			\end{enumerate}\\
			
			\hline
		\end{tabular}
	}
\end{table}


\end{enumerate}
\subsection{\sffamily Performance Requirements}

\subsection{\sffamily Design Constraints}
Follows a set of constraints on the system design imposed by external standards, regulatory requirements, or project
limitations.

\subsubsection{\sffamily Standards compliance}
The application only requires adhesion to the Transport Layer Security standard (TLS) version 1.2 (or higher), \href{https://tools.ietf.org/html/rfc5246}{RFC 5246}
\subsubsection{\sffamily Hardware limitations}
As far as the app version is concerned, CLup \textbf{requires} a device with the following hardware to be operational:\newline
\begin{itemize}
    \item GNSS (any)
    \item screen
    \item pointing input device, such as touchscreens or mice
    \item internet connectivity
    \item audio output device
\end{itemize}

\bigskip \noindent Concerning the \textbf{reduced version} to be run on totems outside the stores, the following hardware is \textbf{requested}:\newline
\begin{itemize}
    \item screen
    \item pointing input device, such as touchscreens or mice
    \item internet connectivity
    \item printer device
\end{itemize}

\bigskip \noindent Finally, the web application for internal use requires: \newline
\begin{itemize}
    \item screen
    \item pointing input device, such as touchscreens or mice
    \item internet connectivity
\end{itemize}


\subsubsection{\sffamily Any other constraint}
There are no other constraints.

\subsection{\sffamily Software System Attributes}

\subsubsection{\sffamily Reliability}
Given the presence of 24/7 stores, as additionally specified inside the Alloy Specification sect. (\ref{sect:alloy}), the system must have a guaranteed uptime of 24 hours a day, 7 days a week.
\subsubsection{\sffamily Availability}
CLup targets the broadest group of individuals possible, given it's aimed at use for everyone as in depth explained in the previous sections. This indeed strains the availability constraints over the application which must be:\newline
\begin{itemize}
    \item available to people living inside slow connection areas, thus requiring a minimum 100Kb/s bandwidth to be operational
    \item available to old and/or low-end devices, being operational on a minimum configuration of:
    \begin{itemize}
        \item Android version 5.0 and higher
        \item 1.5 GHz dual core CPUs for Android devices
        \item 100 MBytes of minimum available storage memory for Android devices
        \item 512 MBytes of system memory for Android devices
        \item iOS version 8.0 or higher
    \end{itemize}
\end{itemize}

\subsubsection{\sffamily Security}
CLup does not require best-in-class security measures given the mainstream nature of its main objectives, yet requires standard security measures to guarantee the following requirements:\newline
\begin{itemize}
    \item User data encryption inside the main Database
    \item Connection encryption between clients and servers to guarantee authentication and integrity 
    \begin{itemize}
        \item Transport Layer Security version 1.2 or higher
    \end{itemize}
\end{itemize}

\subsubsection{\sffamily Maintainability}

\subsubsection{\sffamily Portability}
