
\subsection{Purpose}
This document has the purpose of guiding the developer in the realization process of the \textbf{CLup} software, an innovative application to manage queues digitally.\newline
Due to the Coronavirus emergency grocery shopping needs to follow strict rules: supermarkets need to restrict access to their stores which typically results in long lines forming outside.The goal of this project is to develop an easy-to-use application that allows store managers to regulate the influx of people and that saves people from having to crowd outside of stores.\newline
The application releases a number that gives the position in the queue and gives information about the time when that number is called, in this way the user is able to arrive to the supermarket and enter immediately. \newline
Clup allows also the user to book a slot to enter the supermarket indicating the expected time to shop, or alternatively the application itself can infer it.\newline
Finally the application can suggest different slots to visit the store, based on stores crowdedness, and slots in alternative stores, based on the day/hour preferences of the user.   

\subsubsection{Goals}

\begin{table} [H]
\rowcolors{1}{Goldenrod}{white}
\begin{tabular}{l|l}
	G1 & \pbox{13cm}{Anybody is guaranteed possibility to make shopping at any supermarket in reasonable time (def. reasonable)}\\
	G2 & Users can get to know the least crowded time slots\\
	G3 & Fair users can make a reservation to enter in a supermarket\\
	G4 & Stores can easily monitor fluxes\\
	G5 & Only authorized users can access\\
	G6 & Crowds are dramatically reduced outside supermarket stores\\
    G7 & \pbox{13cm}{CLup should not decrease customer affluence beyond a reasonable level w.r.t. to normal (→ define reasonable)}\\
    G8 & Same shopping capabilities guaranteed to offline users\\
    G9 & \pbox{13cm}{Find the best (less crowded, soonest available) alternative among local supermarket stores (of same franchise only?)}\\
    G10 & Supermarkets do not overcrowd
\end{tabular}
\caption{Goal list}
\label{tab:GoalList}
\end{table}

\subsection{Scope}


\subsubsection{World Phenomena}

\begin{table}[H]
\rowcolors{1}{Goldenrod}{white}
\begin{tabular}{l|l}
	WP1 & User leaves home to go to the supermarket\\\hline
	WP2 & Users crowd outside the store\\\hline
    WP3 & User arrives at the supermarket\\\hline
    WP4 & User enters the supermarket \\\hline
	WP5 & User does the grocery shopping \\\hline
	WP6 & User exits the supermarket\\\hline
	WP7 & Supermarkets restrict accesses in stores\\\hline %do we need this?
	WP8 & User buys products of a non booked category\\\hline %is this a world or shared phenomena?
\end{tabular}
\caption{World phenomena list}
\label{tab:WorldPhen}
\end{table}

\subsubsection{Shared Phenomena}
\begin{table} [H]
\rowcolors{1}{Goldenrod}{white}
\begin{tabular}{l|l}
	SP1 & User lines up using the application \\
	SP2 & User makes a reservation \\
	SP3 & User keeps track of how line evolves \\
	SP4 & User validates the entrance with a QR code \\
	SP5 & User receives suggestion for less crowded time slots \\
	SP6 & User receives suggestion for less crowded stores \\
	SP7 & CLup assigns a time slot \\
	SP8 & CLup signals max number of customers inside the store has been reached \\
	SP9 & CLup signals customer for improper behavior \\
	SP10 & Offline customer interacts with physical totem \\
	SP11 & User confirms booking \\
	SP12 & User confirms ticket reservation \\
	
\end{tabular}
\caption{Shared phenomena list}
\label{tab:SharedPhen}
\end{table}
\subsection{Definitions, Acronyms, Abbreviations}

\subsubsection{Definitions\label{subsub:definitions}}
\begin{itemize}
\item \textbf{Check-in procedure}: the process of getting inside the store. It starts from when the user approaches the entrance, includes the QR ticket scan and ends as soon as the turnstile is passed.
\item \textbf{Reserve entrance}: the process of booking a future entrance (starting from the day next to the current one)
\item \textbf{Malicious user}: someone committed for any reason to CLup malfunction and/or unavailability, shopping disservices.
\item \textbf{Quick ticket}: the actual ticket granting access to the stores. We call it \guillemotleft quick\guillemotright \space to emphasize the difference between booking and lining up.
\item \textbf{Totem}: a desktop based PC with advanced input functionalities (touchscreen), with external hard shell protection, stand mount, (optional) integrated printer.
\item \textbf{Big screen}: a huge screen panel to be located outside the store, in visible placement, used for announcements to offline customers.
\item \textbf{CLup core system\label{core_functionality}}: CLup innermost back-end functionality providing queueing control, access to already enqueued users, access control, big screens operativeness                                                                                                                                                                                                   \end{itemize}


\subsubsection{Acronyms}
\begin{itemize}
\item \textbf{EWT}: Expected Waiting Time
\item \textbf{ASAP}: As Soon As Possible
\item \textbf{WRT}: With Respect To
\end{itemize}

\subsubsection{Abbreviations}


\subsection{Revision History}
\begin{itemize}
	\item \textbf{v1.0}: First version of the document
	\item \textbf{v1.1}: Section 4 revision
	\item \textbf{v1.2}: Sections 3.3, 3.4 revision
	\item \textbf{v1.3}: Section 4.2 revision, sequence diagrams upscaling
	
\end{itemize}


\subsection{Reference Documents}

\subsection{Document Structure}
COPY PASTED FROM RASD TO ANALYSE
\begin{itemize}

\item
Chapter 1 gives an introduction about the purpose of the document and the development of the
application, with its corresponding specifications such as the definitions, acronyms, abbreviation,
revision history of the document and the references.
Besides, are specified the main goals, world and shared phenomena of the software.

\item
Chapter 2 contains the overall description of the project. In the product perspective are included
the statecharts of the major function of the application and the model description through a Class
diagram. In user characteristic are explained the types of actors that can use the application.
Moreover, the product function clarified the functionalities of the application. Finally, are included
the domain assumption that can be deducted from the assignment.

\item
Chapter 3 presents the interface requirement including: user, hardware, software and
communication interfaces. This section contains the core of the document, the specification of
functional and non-functional requirements. Functional requirements are submitted with a list of
use cases with their corresponding sequence diagrams and some scenarios useful to identify
specific cases in which the application can be utilised. Non-functional requirements included:
performance, design and the software systems attributes.

\item
Chapter 4 includes the alloy code and the corresponding metamodels generated from it, with a
brief introduction about the main purpose of the alloy model.

\item
Chapter 5 shows the effort spent for each member of the group.

\item
Chapter 6 includes the reference documents
\end{itemize}

%what you write here is a comment that is not shown in the final text
