\subsection{Purpose}
The purpose of this document is to provide more technical and detailed information about the software
discussed in the RASD document. It will represent a strong guide for the programmers that will develop the
application considering its different parts: the basic service and the two advanced functions.
In this DD we present hardware and software architecture of the system in terms of components and interactions among those components. Furthermore, this document describes a set of design characteristics required for the implementation by introducing constraints and quality attributes.
It also gives a detailed presentation of the implementation plan, integration plan and the testing plan.
In general, the main different features listed in this document are:
\begin{itemize}
	\item The high-level architecture of the system
	\item Main components of the system
	\item Interfaces provided by the components
	\item Design patterns adopted
\end{itemize}
Stakeholders are invited to read this document in order to understand the characteristics of the project
being aware of the choices that have been made to offer all the functionalities also satisfying the quality requirements.

\subsection{Scope}
Clup is an application that aims to avoid users from crowding outside supermarkets when doing grocery shopping in pandemic times.\\
The application can be used both by store customers and store managers. On one hand users can virtually queue by Clup to enter the supermarket and they are provided with real time information about the line, in this way they can arrive at the entrance only when they are allowed to enter. On the other hand the application monitors and stores the information about people fluxes; this data is then provided to store managers who can take actions depending on the situation. 
The few paragraphs just read represent an overview of the main functionalities offered by the system:
more detailed information can be found on the RASD document.
\subsection{Definitions, Acronyms, Abbreviations}
\subsubsection{Definitions\label{subsub:definitions}}
[As per the RASD document]\newline
\begin{itemize}
\item \textbf{The Developers}: we therewith refer to future developers of the CLup application, main target of the present and \textbf{CLup: RASD} documents.
\item \textbf{Check-in procedure}: the process of getting inside the store. It starts from when the user approaches the entrance, includes the QR ticket scan and ends as soon as the turnstile is passed.
\item \textbf{Reserve entrance}: the process of booking a future entrance (starting from the day next to the current one)
\item \textbf{Malicious user}: someone committed for any reason to CLup malfunction and/or unavailability, shopping disservices.
\item \textbf{Quick ticket}: the actual ticket granting access to the stores. We call it \guillemotleft quick\guillemotright \space to emphasize the difference between booking and lining up.
\item \textbf{Totem}: a desktop based PC with advanced input functionalities (touchscreen), with external hard shell protection, stand mount, (optional) integrated printer.
\item \textbf{Big screen}: a huge screen panel to be located outside the store, in visible placement, used for announcements to offline customers.
\item \textbf{CLup core system\label{core_functionality}}: CLup innermost back-end functionality providing queueing control, access to already enqueued users, access control, big screens operativeness
\item \textbf{Affected customers\label{affected:def}}: customer being main or side target of an event or action taking place inside grocery shopping and CLup's scope.
\end{itemize}
\subsubsection{Acronyms}\label{def:acronyms}
\begin{itemize}
\item \textbf{EWT}: Expected Waiting Time
\item \textbf{ASAP}: As Soon As Possible
\item \textbf{WRT}: With Respect To
\item \textbf{PUIF}: Presentation \& User Interaction Functionalities
\item \textbf{FMLBF}: Flow Management \& Local Business Functionalities
\end{itemize}

\subsubsection{Abbreviations}
\subsection{Revision History}
\begin{itemize}
	\item \textbf{v1.0}: First version of the document
	\item \textbf{v1.1}: Component revision, general redesign
	\item \textbf{v1.2}: Architectural changes, final architecture definition
	\item \textbf{v1.3}: Sequence diagrams additions
	\item \textbf{v1.4}: Deployment definition
	\item \textbf{v1.5}: Style and formatting changes
	\item \textbf{v1.6}: Minor revisions and cover addition
	\item \textbf{v1.7}: Definitions, acronyms, abbreviations
	\item \textbf{v1.8}: Acronyms additions, section 2.7 revision, sequence diagrams revision
	\item \textbf{v1.9}: Deployment section revision, sequence diagrams descriptions revision.
	
	
\end{itemize}

\subsection{Reference Documents}
This documents has been designed by taking the \textbf{IEEE Std 1016TM-2009} as primary reference.\cite{STANDARD}\newline


\noindent Other documents of sources have been cited along the way whenever needed.\newline
For additional information over requirements and goals found in this Document, please have a look at the \begin{large}\textbf{Requirements Analysis and Specifications Document (RASD)}                                                                                                                                                                         \end{large} for the CLup application
\subsection{Document Structure}
\begin{itemize}
	\item Chapter 1 describes the scope and purpose of the DD, including the structure of the document and the set of definitions, acronyms and abbreviations used.
	\item Chapter 2 contains the architectural design choice, it includes all the components, the interfaces, the technologies (both hardware and software) used for the development of the application. It also includes the main functions of the interfaces and the processes in which they are utilised (Runtime view and component interfaces). Finally, there is the explanation of the architectural patterns chosen with the other design decisions.
	\item Chapter 3 shows how the user interface should be on the mobile and web application.
	\item Chapter 4 describes the connection between the RASD and the DD, showing the matching between the goals and requirements described previously with the elements which compose the
architecture of the application.
	\item Chapter 5 traces a plan for the development of components to maximize the efficiency of the developer team and the quality controls team. It is divided in two sections: implementation and integration. It also includes the testing strategy.
	\item Chapter 6 shows the effort spent for each member of the group.
	\item Chapter 7 includes the reference documents.
\end{itemize}
